\documentclass{beamer}

\mode<presentation> {
%\usetheme{default}
%\usetheme{AnnArbor}
%\usetheme{Antibes}
%\usetheme{Bergen}
%\usetheme{Berkeley}
%\usetheme{Berlin}
%\usetheme{Boadilla}
%\usetheme{CambridgeUS}
%\usetheme{Copenhagen}
%\usetheme{Darmstadt}
%\usetheme{Dresden}
%\usetheme{Frankfurt}
%\usetheme{Goettingen}
%\usetheme{Hannover}
%\usetheme{Ilmenau}
%\usetheme{JuanLesPins}
%\usetheme{Luebeck}
\usetheme{Madrid}
%\usetheme{Malmoe}
%\usetheme{Marburg}
%\usetheme{Montpellier}
%\usetheme{PaloAlto}
%\usetheme{Pittsburgh}
%\usetheme{Rochester}
%\usetheme{Singapore}
%\usetheme{Szeged}
%\usetheme{Warsaw}

% As well as themes, the Beamer class has a number of color themes
% for any slide theme. Uncomment each of these in turn to see how it
% changes the colors of your current slide theme.

%\usecolortheme{albatross}
%\usecolortheme{beaver}
%\usecolortheme{beetle}
%\usecolortheme{crane}
%\usecolortheme{dolphin}
%\usecolortheme{dove}
%\usecolortheme{fly}
%\usecolortheme{lily}
%\usecolortheme{orchid}
%\usecolortheme{rose}
%\usecolortheme{seagull}
%\usecolortheme{seahorse}
%\usecolortheme{whale}
%\usecolortheme{wolverine}

%\setbeamertemplate{footline} % To remove the footer line in all slides uncomment this line
%\setbeamertemplate{footline}[page number] % To replace the footer line in all slides with a simple slide count uncomment this line

%\setbeamertemplate{navigation symbols}{} % To remove the navigation symbols from the bottom of all slides uncomment this line
}

\usepackage{graphicx} % Allows including images
\usepackage{booktabs} % Allows the use of \toprule, \midrule and \bottomrule in tables
\usepackage{algpseudocode}
\usepackage{pifont}
\usepackage[]{algorithm2e}

%----------------------------------------------------------------------------------------
%	TITLE PAGE
%----------------------------------------------------------------------------------------

\title[Short title]{Wolf and Hare} % The short title appears at the bottom of every slide, the full title is only on the title page

\author{Nils Kohl, Thomas Stadelmayer, David Uhl} % Your name
\institute[FAU] % Your institution as it will appear on the bottom of every slide, may be shorthand to save space
{
Friedrich-Alexander Universit\"at Erlangen-N\"urnberg \\ % Your institution for the title page
\medskip
}
\date{03.07.2015} % Date, can be changed to a custom date

\begin{document}

\begin{frame}
\titlepage % Print the title page as the first slide
\end{frame}

\begin{frame}
\frametitle{Overview} % Table of contents slide, comment this block out to remove it
\tableofcontents % Throughout your presentation, if you choose to use \section{} and \subsection{} commands, these will automatically be printed on this slide as an overview of your presentation
\end{frame}

%----------------------------------------------------------------------------------------
%	PRESENTATION SLIDES
%----------------------------------------------------------------------------------------

%------------------------------------------------
\section{Wolf and Hare}
\subsection{Spielregeln}
\begin{frame}
\frametitle{Spielregeln}
\begin{itemize}
\item 2D Spielfeld mit zwei W\"olfen und einem Hasen
\item Zuf\"allige Startposition auf Spielfeld
\item Pro Zug: W\"olfe jeweils einen Schritt, Hase einen Schritt
\item Spielende: Wolf f\"angt Hasen oder Hase erreicht Ziel
\end{itemize}
\begin{block}{Aufgabe}
\begin{itemize}

\item Spiel parallelisieren
\item Jede Maschine auf Cluster testet verschiede Routen
\end{itemize}

\end{block}
\end{frame}
 
%------------------------------------------------
\subsection{Visualisierung}
\begin{frame}
\frametitle{Visualisierung (1)}
\begin{columns}[c] % The "c" option specifies centered vertical alignment while the "t" option is used for top vertical alignment

\column{.45\textwidth} % Left column and width
\begin{table}
\begin{tabular}{l l}
\toprule
\textbf{x-Position} & \textbf{y-Position}\\
\midrule
 0 & 0 \\
0 & 1 \\
0 & 2 \\
0 & 3 \\
0 & 4 \\
1 & 4 \\
2 & 4 \\
3 & 4 \\
4 & 4 \\
\bottomrule
\end{tabular}
\caption{Wolf1}
\end{table}

\column{.5\textwidth} % Right column and width
Bild der Route einfuegen
\end{columns}

\end{frame}

\begin{frame}

\frametitle{Visualisierung (2)}
\begin{columns}[c] % The "c" option specifies centered vertical alignment while the "t" option is used for top vertical alignment

\column{.45\textwidth} % Left column and width
\begin{table}
\begin{tabular}{l l}
\toprule
\textbf{x-Position} & \textbf{y-Position}\\
\midrule
 0 & 0 \\
1 & 0 \\
2 & 0 \\
3 & 0 \\
4 & 0 \\
4 & 1 \\
4 & 2 \\
4 & 3 \\
4 & 4 \\
\bottomrule
\end{tabular}
\caption{Route1 von Wolf1}
\end{table}

\column{.5\textwidth} % Right column and width
\begin{table}
\begin{tabular}{l l}
\toprule
\textbf{x-Position} & \textbf{y-Position}\\
\midrule
 0 & 2 \\
0 & 3 \\
0 & 4 \\
1 & 4 \\
2 & 4 \\
3 & 4 \\
4 & 4 \\
4 & 4 \\
4 & 4 \\
\bottomrule
\end{tabular}
\caption{Route1 von Hase}
\end{table}
\end{columns}

\end{frame}

\begin{frame}
\frametitle{Visualisierung (3)}

\end{frame}

\section{Implementierung}
\subsection{Seriell}

\begin{frame}
\frametitle{Seriell}
\begin{itemize}
\item Vergleiche Route1 von Wolf1 einer Routen vom Hasen
\item In neue Liste Anzahl der Schritte bis Hase gefangen wird \\(-1 wenn Hase vorher im Ziel)
\item Alle Routen von Wolf1 
\item Analog Wolf2
\item Vergleiche Liste von Wolf1 und Liste von Wolf2
\item Ausgabe Beste Routen
\end{itemize}
\end{frame}

\begin{frame}
\begin{algorithm}[H]
\frametitle{Seriell}
% \KwData{this text}
% \KwResult{how to write algorithm with \LaTeX2e }
% initialization\;
\For{all wolf1 routes} {
 \For{all hare routes}{
  compare route\_i of wolf1 with route\_j of hare\;
  \Comment{compares each element}\;
  \Comment{counter++}\;

  \eIf{capture}{
   push counter to list\;
   }{
   add -1 in list\;
  }
 }
 }
 \caption{How to get best routes}
\end{algorithm}
\end{frame}

\subsection{Parallel}
\begin{frame}
\frametitle{Parallel}
\begin{itemize}
\item Tupel von einer Route Wolf1 und Wolf2
\item Vergleiche Tupel mit allen Routen von Hase
\item Schreibe in Liste Anzahl der Schritte und Wahrscheinlichkeit als Indikatoren
\item Wenn fuer neues Tupel bessere Loesung gibt, dann loesche alten Wert und schreibe neuen rein

\item Tupel auf Prozessoren aufteielen
\item Nach Berechnung vergleiche die Listen
\end{itemize}
\end{frame}

\begin{frame}
\begin{algorithm}[H]
\frametitle{Parallel}
% \KwData{this text}
% \KwResult{how to write algorithm with \LaTeX2e }
% initialization\;
 $<$r1w1,r2w2$>$  = choose tupel of routes wolf1 and wolf2\;
 \For{all hare routes}{
  compare tupel with route\_i of hare\;
  \Comment{compares each element}\;
  \Comment{counter++ and clac probability}\;

  \eIf{capture}{
	\If{counter $<$ counter of prev list\_elem}{
		   replace tupel of counter and probability ($<$ 5,$\dfrac{1}{24}$ $>$) in list\;
	}  
	\If{counter $=$ counter of prev list\_elem}{
		push tupel of counter and probability to list\;	
	}
  
   }{
	continue\;
  }
 }
 \caption{How to get best routes}
\end{algorithm}
\end{frame}


\end{document} 